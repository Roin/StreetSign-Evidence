\chapter{Einführung}
Anders als im Readme beschrieben haben wir uns für einen
einzigen File als Input entschieden. Der Gründe dafür waren
das im ersten Beispiel nur eine einzige Datei vorlag und wir
deshalb unseren ursprünglichen Entwurf darauf ausgelegt hatten.
Außerdem spielt es auch keine Rolle ob man ein oder zwei
Dateien verwendet.\\
\section{Input File Format}
  Das Input-File ist in 2 Sektionen unterteilt, welche mittels
  Semikolon voneinander getrennt sind. Die einzelnen Attribute
  werden mittels Komma separiert.\\
  Aus diesem Format sind wir nun in der Lage Primitive und
  Relationen zu generieren.
  Zuerst werden die Primitive ausgelesen und in eine geeignete
  Datenstruktur gespeichert, anschließend die Relationen.