\chapter{Verfahren}
In diesem Kapitel wird das verwendete Vefahren noch einmal
theoretisch beleuchtet.
\section{Dempster-Shafer Theorem}
  Während des Auslesens des Input Files wird ein HashTable
  der einzelenen Primitiven gebildet. Sollte über ein Primitiv 
  mehrere Aussagen vorliegen so wird nun zuerst die Gesamtevidenz
  über dieses Primitiv gebildet. Die Ausgangslage zur 
  Evidenzbestimmung ist die Zuverlässigkeit der Aussage über
  ein Primitiv (Basismass). Nun wird für jede "inside" 
  Relation die Evidenz mittels der Dempster-Shafer Regel
  zusammen gefasst, also für Objekte mit mehreren Aussagen 
  die Gesamtevidenz und für einzelne die gegebene 
  Zuverlässigkeit.\\
  Sollte es dabei zu einem Konflikt kommen
  wird zusätzlich noch ein Korrekturfaktor bestimmt. 
  Anschließend wird der Believe über die verschiedenen
  Straßenschilder gebildet, Grundlage dafür sind die in der
  Aufgabenstellung beschriebenen Straßenschilder und die im
  Input File gegeben Relationen. \\
  Anschließend wird noch bestimmt ob gewisse Schilder übereinander
  ("ontop" liegen können. Dazu werden die ontop Relationen
  geprüft und die Objekte die bereits als Schild idendifiziert wurden
  zusammen genommen. Mittels Dempster-Shafer Regel wird nun die Evidenz
  für die Relation bestimmt und falls vorhanden die Relation ausgegben,
  also welches Schild über welchem Schild liegt.
  Abschließend wird die Plausabilität, ob wir ein Schild
  finden können zusammen mit dem Believe das wir ein Schild
  gefunden haben ausgegeben.
