%
% Nahezu alle Einstellungen koennen hier getaetigt werden
%

\documentclass[%
	pdftex,
	oneside,		% Einseitiger Druck.
	12pt,			% Schriftgroesse
	parskip=half,	% Halbe Zeile Abstand zwischen Absätzen.
	headsepline,	% Linie nach Kopfzeile.
	footsepline,	% Linie vor Fusszeile.
	abstracton,	    % Abstract Überschriften
	ngerman,		% Translator
]{scrreprt}

%Seitengroesse
\usepackage{fullpage}

%Zeilenumbruch und mehr
\usepackage[activate]{microtype}

% Zeichencodierung
\usepackage[utf8]{inputenc}
\usepackage[T1]{fontenc}

% Zeilenabstand
\usepackage[onehalfspacing]{setspace}

% Index-Erstellung
\usepackage{makeidx}

% Lokalisierung (neue deutsche Rechtschreibung)
%\usepackage[english]{babel}
\usepackage[ngerman]{babel} 

% Anführungszeichen 
\usepackage[babel,german=quotes]{csquotes}
%\usepackage[style=swiss]{csquotes}


% Spezielle Tabellenform fuer Deckblatt
\usepackage{longtable}
\setlength{\tabcolsep}{10pt} %Abstand zwischen Spalten
\renewcommand{\arraystretch}{1.5} %Zeilenabstand

% Grafiken
\usepackage{graphicx}

% Mathematische Textsaetze
%\usepackage{amsmath}
%\usepackage{amssymb}

% Pakete um Textteile drehen zu können, oder eine Seite Querformat anzeigen kann.
%\usepackage{rotating}
%\usepackage{lscape}

% Farben
\usepackage{color}
\definecolor{LinkColor}{rgb}{0,0,0.2}
\definecolor{ListingBackground}{rgb}{0.92,0.92,0.92}

% PDF Einstellungen
\usepackage[%
	pdftitle={\pdftitel},
	pdfauthor={\autor},
	pdfsubject={\arbeit},
	pdfcreator={pdflatex, LaTeX with KOMA-Script},
	pdfpagemode=UseOutlines, % Beim Oeffnen Inhaltsverzeichnis anzeigen
	pdfdisplaydoctitle=true, % Dokumenttitel statt Dateiname anzeigen.
	pdflang=en % Sprache des Dokuments.
]{hyperref}

% (Farb-)einstellungen für die Links im PDF
\hypersetup{%
	colorlinks=false, % Aktivieren von farbigen Links im Dokument
	linkcolor=LinkColor, % Farbe festlegen
	citecolor=LinkColor,
	filecolor=LinkColor,
	menucolor=LinkColor,
	urlcolor=LinkColor,
	bookmarksnumbered=true % Überschriftsnummerierung im PDF Inhalt anzeigen.
}

% Verschiedene Schriftarten
%\usepackage{goudysans}
%\usepackage{lmodern}
%\usepackage{libertine}
\usepackage{palatino} 

% Hurenkinder und Schusterjungen verhindern
% http://projekte.dante.de/DanteFAQ/Silbentrennung
\clubpenalty=10000
\widowpenalty=10000
\displaywidowpenalty=10000

% Quellcode
\usepackage{listings}
\lstloadlanguages{Java}
\lstset{%
	language=Java,		 	 % Sprache des Quellcodes
	numbers=left,           % Zelennummern links
	stepnumber=1,            % Jede Zeile nummerieren.
	numbersep=5pt,           % 5pt Abstand zum Quellcode
	numberstyle=\tiny,       % Zeichengrösse 'tiny' für die Nummern.
	breaklines=true,         % Zeilen umbrechen wenn notwendig.
	breakautoindent=true,    % Nach dem Zeilenumbruch Zeile einrücken.
	postbreak=\space,        % Bei Leerzeichen umbrechen.
	tabsize=2,               % Tabulatorgrösse 2
	basicstyle=\ttfamily\footnotesize, % Nichtproportionale Schrift, klein für den Quellcode
	showspaces=false,        % Leerzeichen nicht anzeigen.
	showstringspaces=false,  % Leerzeichen auch in Strings ('') nicht anzeigen.
	extendedchars=true,      % Alle Zeichen vom Latin1 Zeichensatz anzeigen.
	captionpos=b,            % sets the caption-position to bottom
	backgroundcolor=\color{ListingBackground} % Hintergrundfarbe des Quellcodes setzen.
}
% Deutsche Überschriften im Appendix
\addto{\captionsngerman}{
  \renewcommand{\contentsname}{\sffamily Inhaltsverzeichnis}
  \renewcommand{\listfigurename}{\sffamily Abbildungsverzeichnis}
  \renewcommand{\lstlistingname}{\sffamily Quellcodeverzeichnis}
  \renewcommand{\lstlistlistingname}{\sffamily Quellcodeverzeichnis}
}


% Glossar
\usepackage[
	nonumberlist, %keine Seitenzahlen anzeigen
	acronym,      %ein Abkürzungsverzeichnis erstellen
	section,      %im Inhaltsverzeichnis auf section-Ebene erscheinen
	toc,          %Einträge im Inhaltsverzeichnis
]{glossaries}

% Fussnoten
\usepackage[perpage, hang, multiple, stable]{footmisc}

% Seitenränder
\usepackage{anysize}
\marginsize{30mm}{25mm}{25mm}{25mm}

% Titel, Autor und Datum
\title{\titel}
\author{\autor}
\date{\datum}
